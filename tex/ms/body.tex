\section{Introduction}

The Dirichlet process (DP) is specified by a based distribution \basedistribution
and a concentration parameter \concentration parameter.
For now, let's ignore \basedistribution and only focus on the partitioning
aspect of the Dirichlet process, which is controlled by \concentration.
That is, if we are assigning \numelements elements to an unknown number of
categories, the \concentration determines the probability of all possible set
partitions of the \numelements.

When assigning the the next element, it goes into
a new category with probability
\begin{equation}
    \frac{\concentration}{\concentration + \numassignedtotal},
    \label{eq:dpprobnew}
\end{equation}
where \numassignedtotal is the number of elements that have already been
assigned.
The element can also be assigned to an existing category \catindex with probability
\begin{equation}
    \frac{\numassigned[\catindex]}{\concentration + \numassignedtotal},
    \label{eq:dpprobexisting}
\end{equation}
where \numassigned[\catindex] is the number of elements already assigned
to category \catindex.

The Pitman-Yor process (PYP) generalizes the Dirichlet process by adding
a discount parameter \discount.
Under a Pitman-Yor process
we assign the next element to a new category with probability
\begin{equation}
    \frac{\concentration + \numcats\discount}{\concentration + \numassignedtotal},
    \label{eq:pyprobnew}
\end{equation}
where \numcats is the number of existing categories.
We assign the next element to an existing category with probability
\begin{equation}
    \frac{\numassigned[\catindex] - \discount}{\concentration + \numassignedtotal}.
    \label{eq:pyprobexisting}
\end{equation}
This creates a penalty for adding an element to an existing category, but as we
can see in Equation~\ref{eq:pyprobexisting}, this penalty has a greater effect
on categories with fewer elements.
Thus, conditional on adding an element to an existing category, this
excacerbates the ``rich-get-richer'' property of the Dirichlet process.
Furthermore, the relative probability of assigning a new element to existing
large categories increases as the penalty (\discount) increases (conditional on
assigning it to an existing category).

\subsection{Weighted-discount process}

To give small existing categories more probability of getting more elements
compared to large existing categories, we can weight the discount parameter
by the number of elements in each category. Specifically, 
we can assign the next element to a new category with probability
\begin{equation}
    \frac{\concentration + \numassignedtotal\discount}{\concentration + \numassignedtotal},
    \label{eq:wdprobnew}
\end{equation}
and to an existing category with probability
\begin{equation}
    \frac{\numassigned[\catindex] - (\numassigned[\catindex]\discount)}{\concentration + \numassignedtotal}.
    \label{eq:wdprobexisting}
\end{equation}
Now the penalty is proportional to the size of each category.
We'll call this the weighted-discount process (WDP).

This has some interesting properties.
It still converges to the Dirichlet process when $\discount = 0$ (just like the
Pitman-Yor) process.
However, now there are two ways the process can converge to a continuous
process (i.e., the probability that number of categories equal the number of
elements becomes one); the \concentration can go to infinity or the
\discount can approach 1.
This is because the WDP is a mixture (\jrocomment{I'm not sure if ``mixture''
    is technically correct here}) of two processes: the Dirichlet process and a
pure weighted-discount process.
When the \discount is zero, we get the Dirichlet process.
As the \concentration approaches zero, we converge on a pure
weighted-discount process, where the next element is assigned to
a new category with probability
\begin{equation}
    \frac{\numassignedtotal\discount}{\numassignedtotal},
    \label{eq:pwdprobnew}
\end{equation}
and to an existing category with probability
\begin{equation}
    \frac{\numassigned[\catindex] - (\numassigned[\catindex]\discount)}{\numassignedtotal}.
    \label{eq:pwdprobexisting}
\end{equation}
In this special case the \discount parameter becomes the probability of
assigning an element to a new category.


\subsection{An alternative weighted-discount process}

We can also think of adjusting the weights such that the absolute weight of
assigning to new category is the same as in the Pitman-Yor process.
The next element is assigned to
a new category with probability
\begin{equation}
    \frac{\concentration + \numcats\discount}{\concentration + \numassignedtotal},
    \label{eq:pywdprobnew}
\end{equation}
and to an existing category with probability
\begin{equation}
    \frac{\numassigned[\catindex] - (\frac{\numassigned[\catindex]}{\numassignedmean}\discount)}{\concentration + \numassignedtotal},
    \label{eq:pywdprobexisting}
\end{equation}
where \numassignedmean is the mean number of elements already assigned to
categories.  We still have the interval of the discount of $\discount \in [0,
1)$, because the smallest possible \numassignedmean is 1.
This is similar to the Pitman-Yor process, but with some ``smoothing'' out of
the relative probabilities of existing categories getting a new element.
This avoids the process having two ways of converging to continuous
(i.e., the probability that number of categories equal the number of
elements becoming one).
